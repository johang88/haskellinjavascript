\section{Slutsatser}

\subsection{Framtida förbättringar}


\subsubsection{Parser}
Hjälpsamma och förklarande felmeddelanden är en viktigt del av ett utvecklingsverktyg och genereras för tillfället inte av parsern. 
Om parsern stöter på ett fel rapporterar den att ett fel har inträffat och avslutar parsningsprocessen. 
Att förbättra dessa felmeddelanden med exempelivs rad- och kolumnnummer och specifik information om vad för fel som har inträffat skulle göra parsern mer användbar.

För att implentera detta behöver steg 1 och 2 kombiners för att rad- och kolumn-nummer ska bevaras korrekt då borttagning av nästlade kommentarer kan påverka dessa. 
JSParse behöver modifieras så att det rapporterar var ett fel uppstod och i vilken parser.

Konverteringen av icke kontextfri Haskellkod klarar inte av att expandera måsvingar i \empth{[x | let x = 5]},
för att lösa det behöver parsern modifieras så att den räknar antalet måsvingar, paranteser, komman och hakparanteser efter
\emph{let} och avgör när det är korrekt att sätta in avlsutande måsvingar.

\subsubsection{HIJi}
HIJi är tänkt som ett webbaserat verktyg för nybörjare att komma igång med funktionell programmering. Ur det perspektivet når HIJi ännu inte upp till de krav som en nybörjare kan förvänta sig. Avsaknaden av interaktivitet är den i dagsläget största nackdelen. Här är de förbättringar som vi anser vara nödvändiga för att nybörjare ska kunna använda sig utav HIJi.

Erbjuda en interaktiv tutorial där användaren får instruktioner vad som ska skrivas in i HIJi. Om använaren skriver in rätt uttryck fortsätter tutorialen till nästa nivå.

Få ut typinformation ur funktioner genom att hålla musen över funktionsnamnet. 
