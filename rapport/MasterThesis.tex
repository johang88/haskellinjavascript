% USE PDFLATEX WITH THIS DOCUMENT. If you wish to use ordinary latex (why?) then convert the figures in figures/ to eps.
% If you are using Chalmers linux system, you might need to change to a newer tex dist.
% Do so with: vcs-what latex 
% look up the newest version and select it, a.t.m that would be: vcs-select -p texlive-20080816
% I recommend using kile in linux. Make it available by adding the unsup software 
% echo unsup ~/.vcs4/pathsetup
\documentclass[a4paper,12pt]{article}

\usepackage[T1]{fontenc}
\usepackage[swedish]{babel}
%\usepackage[latin1]{inputenc} % Use the same as the encoding of the textfile
\usepackage[utf8]{inputenc} % But this is probably the best choice. This doument is currently in utf8 (so is all other files)
\usepackage{amsmath,amsfonts,amsthm,mathtools} % Math. Look up the amsmath manual (google) for many great examples.
\usepackage{graphicx,placeins,float} % Figures and placements, float must be loaded before hypperref
\usepackage[colorlinks,citecolor=blue]{hyperref} % Warning this package conflicts with the package algorithm
\usepackage{fancyhdr,url,tikz} % Other usefull stuff. tikz is good for making drawings (but difficult)
\usepackage{listings} % Source code
% \usepackage{a4wide} % Gives a wider page
\usepackage{mycommands} % Here you can define your personal favourites. See mycommands.cls
\usepackage{pgfplots} % Anoth nice library for plots

\usepackage{natbib} % ger Harvard-referenser


%mina andringar för att få riktiga paragrafer vid radbrytning
\setlength{\parskip}{12pt}
\setlength{\parindent}{0pt}



% mysubtitle, cover* and divisionnameB may be uncommented if nonexistant.
\newcommand{\mytitle}{Haskell in Javascript}
%\newcommand{\mysubtitle}{Some subtitle}
\newcommand{\writtenmonth}{May}
\newcommand{\writtenyear}{2010}
\newcommand{\authors}{Adam Bengtsson\\Mikael Bung\\Johan Gustafsson\\Mattis Jeppsson}
\newcommand{\authorsc}{ Adam Bengtsson, Mikael Bung, Johan Gustafsson, \\Mattis Jeppsson} % again with comma
\newcommand{\YYYYNN}{2010:05}
%\newcommand{\ISSN}{1652-8557}
%\newcommand{\covercaption}{Haskell 98 implementerat i Javascript för att köras i en webbläsare}
% \newcommand{\coverfigure}{figures/COVER_93_iso.png}
\newcommand{\departmentname}{Computer Science and Engineering}
\newcommand{\divisionnameA}{Computer Engineering}
%\newcommand{\divisionnameB}{Fluid Mechanics}
% \newcommand{\nameofprinters}{Chalmers Reproservice}
\newcommand{\nameofprogramme}{Computer Science and Engineering Programme}
\newcommand{\keywords}{Javascript, Haskell, Parser, Interpreter, Type Checker}
% This is just to keep the swedish letters at one place in case there are problems with encoding
\newcommand{\city}{Göteborg} 
\newcommand{\country}{Sweden} 
\newcommand{\university}{Chalmers University of Technology\\ Gothenburg University}
\newcommand{\thesis}{Bachelor's thesis}
\newcommand{\telephone}{+ 46 (0)31-772 1000}
\newcommand{\postcode}{SE-412 96}
% End of input.

% Use section number first in numbering
\numberwithin{equation}{section}
\numberwithin{figure}{section}
\numberwithin{table}{section}

% Setting up the marginals a bit larger
%\textheight=730pt % default 609pt, ~710 if you dont use a4wide
%\headsep=0pt % default 25pt, since we have no headers in this document
%\headheight=0pt% 12 pt
%\voffset=-0.4in % default 0

% Set up headers to match styleguide
\pagestyle{fancy}
\renewcommand{\headrulewidth}{0pt}
\fancyhead{}
\fancyfoot{}
\fancyfoot[C]{\footnotesize \includegraphics[height=2.5mm]{figures/Logo.pdf}, \textit{\departmentname}, \thesis\ \YYYYNN}
\fancyfoot[RO,LE]{\thepage}

% To use code use \lstinputlisting[langauge=matlab,style=mystyle]{somematlabfile.m}
% Here is a suitable style for code.

%\lstset{language=Haskell}
%\lstset{keywordstyle=\color{blue}\textbf} 
%\lstset{stringstyle=\color{red}}

\lstdefinestyle{mystyle}{showstringspaces=false, basicstyle=\scriptsize\ttfamily,
        frame=shadowbox, breaklines=true, numbers=left, commentstyle=\color{gray},
        keywordstyle=\color{blue}\textbf, stringstyle=\color{red}}

% Metadata ion the PDF file (makes it searchable)
\hypersetup{pdfauthor={\authorsc},pdftitle={\mytitle},pdfsubject={\thesis},pdfkeywords={\keywords}}
        
% The following part is automatically generated, go to document start
\begin{document}
\thispagestyle{empty}
\begin{tikzpicture}[remember picture,overlay]
 \node[yshift=-6.4cm] at (current page.north west)
   {\begin{tikzpicture}[remember picture, overlay]
     % First draw the grid and then the Logo and avancez logo.
     \draw[clip] (0cm,6.4cm)--(\paperwidth,6.4cm)--(\paperwidth,0cm)--(0.725\paperwidth,0cm)
               .. controls (0.703\paperwidth,0) and (0.703\paperwidth,0.8cm)..
               (0.68\paperwidth,0.8cm)--(0cm,0.8cm)--cycle;
     \draw[step=5mm,black] (0,0) grid (\paperwidth,6.4cm);
     \node[anchor=west,xshift=2.05cm,yshift=3.27cm,rectangle]{\includegraphics[width=13.2cm]{figures/Logo.pdf}};
     \node[anchor=west,xshift=15.65cm,yshift=3.3cm,rectangle]{\includegraphics[width=3.6cm]{figures/Avancez.pdf}};
 %    \node[anchor=west,xshift=1.15cm,yshift=0.1cm,rectangle]{\includegraphics[width=13.2cm]{figures/gulogga.png}};

     \end{tikzpicture}};
\end{tikzpicture}
\ \vfill
\makeatletter\@ifundefined{coverfigure}{}{
 \begin{center}
  \includegraphics[width=\textwidth,height=0.4\paperheight,keepaspectratio]{\coverfigure}
 \end{center}
}\makeatother
{\huge\noindent \mytitle\par} % title, 21
{\large\noindent \ \par} % subtitle, 16
%{\normalsize\noindent\textit{Master of Science Thesis}\par};
{\normalsize\noindent\textit{\thesis\ in \nameofprogramme}\par} % 14
\vskip 5mm
{\Large\noindent \uppercase\expandafter{\authors}\par}% 18
\vskip 6mm
{\small\noindent
Department of \departmentname\\
\textit{Division of \divisionnameA}
\makeatletter\@ifundefined{divisionnameB}{}{\textit{and Division of \divisionnameB}}\makeatother\\
\uppercase\expandafter{\university}\\
\city, \country\ \writtenyear\\
\thesis\ \YYYYNN\par}

 \newpage
\thispagestyle{empty}
\mbox{}

\newpage
\thispagestyle{empty}
\begin{center}
 {\uppercase\expandafter{\thesis}\ \YYYYNN\par}
 \vskip 40mm
 {\Large \mytitle\par}
 \vskip 5mm
 \makeatletter\@ifundefined{mysubtitle}{}{{\mysubtitle\par}}\makeatother
 \vskip 5mm
 {\thesis\ in \nameofprogramme\par} % (if applicable) (if applicable)
 {\uppercase\expandafter{\authors}\par}
 \vfill
 {Department of Computer Science\par}
 {\textit{Division of \divisionnameA}
  \makeatletter\@ifundefined{divisionnameB}{}{\textit{and Division of \divisionnameB}}\makeatother
 \par}
 {\uppercase\expandafter{\university}\par}
 \vskip 2mm
 {\city, \country\ \writtenyear\par}
\end{center}

\newpage
{\noindent \mytitle\\
\makeatletter\@ifundefined{mysubtitle}{}{\mysubtitle\\}\makeatother
{\uppercase\expandafter{\authors}}\par}
\vskip 10mm
{\noindent \copyright {\uppercase\expandafter{\authorsc}}, \writtenyear\par}
\vskip 20mm
{\noindent \thesis\ \YYYYNN\\
Department of \departmentname\\
\makeatletter\@ifundefined{divisionnameB}{}{ and Division of \divisionnameB}\makeatother
\\
\university\\
\postcode\ \city\\
\country\\
Telephone: \telephone\par}
\vfill
\makeatletter\@ifundefined{covercaption}{}{{\noindent Cover:\\\covercaption\par}\vskip 5mm}\makeatother
{\noindent \nameofprinters 
%/ Department of \departmentname
\\\city, \country\ \writtenyear\par}
\thispagestyle{empty}

\newpage
%\thispagestyle{justpage}
\setcounter{page}{1}
\pagenumbering{Roman}
%{\noindent \mytitle\\
%\makeatletter\@ifundefined{mysubtitle}{}{\mysubtitle\\}\makeatother
%\thesis\ in \nameofprogramme\\ % (if applicable)
%{\uppercase\expandafter{\authors}}\\
%Department of \departmentname\\
%Division of \divisionnameA
%\makeatletter\@ifundefined{divisionnameB}{}{ and Division of \divisionnameB}\makeatother\\
%\university\par}


\hyphenpenalty=10000
\tolerance=5000

%\hyphenation{dereference,heapptr,expression} %, typcheckaren, miss-lyckande, ghci, haskelltolk, haskell, typinformation, javascript, interpreter, programmering, interpretatorn, functions, haskell-kompilator, lambda, modul, exempelvis, grammatik, interpreteringen, pekar, av-sockrar, upp-byggnad, funktions-defenitioner, returnera, av-sockrade, skickas }

\phantomsection\addcontentsline{toc}{section}{Abstract}\renewcommand{\abstractname}{Abstract}
    \begin{abstract}
        This is the abstract. yyp
        This is the abstract. yyp
        This is the abstract. yyp
        This is the abstract. yyp
        This is the abstract. yyp
        This is the abstract. yyp
        This is the abstract. yyp
    \end{abstract}

%\noindent Keywords: \keywords

% Either a swedish translation, or an emtpy page.
\newpage
\phantomsection\addcontentsline{toc}{section}{Sammanfattning}\renewcommand{\abstractname}{Sammanfattning}

\begin{abstract}
  Detta 'är sammanfatntingen
  Detta 'är sammanfatntingen
  Detta 'är sammanfatntingen
  Detta 'är sammanfatntingen
  Detta 'är sammanfatntingen
  Detta 'är sammanfatntingen
\end{abstract}


\newpage
\mbox{}

%\newpage
\phantomsection\addcontentsline{toc}{section}{Innehåll}
\tableofcontents

% Here you can add preface and notations
%\cleardoublepage
%\phantomsection\addcontentsline{toc}{section}{Preface}\input{Preface}
%\vskip 1cm
%\noindent \city\ \writtenmonth\ \writtenyear\\
%\authorsc
%\newpage
%\phantomsection\addcontentsline{toc}{section}{Notations}\input{Notations}

\cleardoublepage
\setcounter{page}{1}
\pagenumbering{arabic}


% Real contents of report starts here
% Splitting it up to several files help when working together.
% Floatbarriers prevent figures from beeing placed into the next chapter.
\section{Inledning}

\subsection{Bakgrund}
På vissa av Chalmers och Göteborgs Universitets datorvetenskapliga program är den första programmeringskursen i Haskell \citep{haskell98} och för en del av de nya eleverna är inlärningströskeln relativt hög. De studenter som börjar på de datavetenskapliga programmen på Chalmers och Göteborgs Universitet är allt från nybörjare till mycket kompetenta inom programmering. Det som dock förenar de flesta är att de saknar kunskaper kring funktionell programmering. De flesta som tidigare har programmerat har gjort det i något av de stora mainstream-programmeringsspråken, vilket i nästan alla fall innebär ett objektorienterat programmeringsspråk. Skillnaden mellan ett funktionellt programmeringsspråk och ett objektorienterat är stora och omställningen hur man behöver angripa programmeringsrelaterade problem är inte enkel för de flesta nybörjare. Vi tror att ett interaktivt webverktyg skulle kunna sänka den här tröskeln och underlätta undervisningen. Ett webverktyg medför även att man slipper installera och lära sig extra verktyg så som Glasgow Haskell Compiler \citep{ghc}. Webbens stöd för interaktivitet gör det möjligt att snabbt visa funktionsdeklarationerna för de inbyggda funktionerna och att enkelt evaluera funktionerna och testa sig fram till olika resultat.

Det är även så att många programmerare inte kommer i kontakt med funktionell programmering  och med hjälp utav ett interaktivt webbverktyg som är enkelt för användaren att använda så är vår förhoppning att fler programmerare och studenter ska komma i kontakt med funktionell programmering, och i synnerhet Haskell. Då flera moderna objektorienterade programmeringsspråk börjar ta begrepp och funktionalitet ifrån funktionella programmeringsspråk så är det extra viktigt att programmerare kommer i kontakt med funktionell programmering Ett exempel på detta är C\# som i senare versioner har fått stöd för bland annat lambdafunktioner \citep{csharp}. 

En stor fördel med att ha tolken på webben är att det enda som behövs för att använda den är en javascriptkompatibel webbläsare, något som följer med i princip i alla moderna operativsystem. Detta betyder att de användare som befinner sig inom vår målgrupp redan har den programvaran som behövs på sina hemdatorer för att använda sig utav vårat program.

% TODO teori eller bakgrund?
Haskell är ett starkt statiskt typcheckat och funktionellt programmeringsspråk med icke-strikt semantik. 
Att språket är funktionellt innebär bland annat att funktioner är \emph{first-class citizens} och kan därmed användas som parametrar och returneras från andra funktioner precis som vilken annan typ som helst.

% TODO teori eller bakkgrund?
Icke strikt semantik, även kallat \emph{lazy evaluation}, innebär mer konkret att evalueringen av ett uttryck inte kommer utföras förrän resultatet av uttrycket behövs. Om uttrycket inte behövs  kommer interpretatorn att ignorera det. 
%skriva in kod XD    \lstinputlisting[language=Haskell, numbers=left]{kapitel/ex1.hs}

Lazy evaluation gör att programmeraren inte behöver bry sig om exekveringsordningen av ett program. Detta ger prestandaförbättringar eftersom ett uttryck inte evalueras alls om det inte behövs \citep{hudak89}.
Lazy evaluation gör det också möjligt att använda sig utav oändliga datastrukturer, till exempel oändliga listor. Språket blir därmed mer uttrycksfullt. 

Funktionella programmeringsspråk såsom Haskell anses också vara det naturliga steget att ta när man vill nå en högre abstraktionsnivå än den som imperativa programmeringsspråk tillåter.
Detta för att funktionella programmeringsspråk tillåter programmeraren att skriva program som är mer modulära, lättare att dela upp i separata delar, än imperativa programmeringsspråk. Den ökade modulariteten beror på att de stödjer tekniker såsom lazy evaluation och higher order functions.
Detta bidrar i sin tur till att program skrivna i Haskell är generellt sätt kortare än ett program skrivet i ett imperativt programmeringsspråk  \citep{why}.

Med ovan nämnda resonemang ser vi det som ovärderligt för programmerare att komma i kontakt och lära sig funktionell programmering. 
Förhoppningen är att vår Haskelltolk i Javascript i första hand ska kunna användas som en interaktiv läroplattform för studenter men även fungera som en inkörsport för programmerare till funktionell programmering. 


\subsection{Syfte}
Syftet är en implementera en fungerande implementation av en haskelltolk i Javascript. Den ska kunna tolka en delmängd utav Haskell-specifikationen så att den kan användas för att göra exempelvis interaktiva tutorials för nybörjare.
Meningen är att dessa ska kunna köras i en vanlig webbläsare utan att ladda ner en haskellkompilator, till exempel GHC, eller behöva lära sig krångliga kommandon.

\subsection{Problem} 

\subsection{Metod}
Det normala tillvägagångssättet när man skriver en tolk är att man först
skapar en parser för den aktuella syntaxen, sedan en typcheckare med 
hjälp av de för språket definierade typereglerna och sist en interpretator
som tolkar språket utefter dess specifikation. Vi hade tänkt följa den här planen genom varje milstolpe genom att utöka parsern, typcheckaren och interpretatorn med ny funktionalitet.

Ett lämpligt delmål är att först göra en enkel implementation utav lambda calculus då mer avancerade funktionella programspråksegenskaper kan implementeras som detta \citep{jones87}.

 Parsern implementeras med hjälp av ett parser combinator bibliotek kallat \emph{JSParse} \citep{jsparse}. Detta ger oss möjlighet att implementera den del av Haskells syntax som inte är context free relativt enkelt.

Vi kommer även att använda det biblioteket för att bygga ett eget syntaxträd som skickas vidare till typcheckaren och interpretatorn I typcheckaren dekoreras syntaxträdet med typinformation.


En interaktiv prompt som kan köras i en webbläsare kommer att utvecklas. Den ska ge användaren möjlighet att skriva Haskell-funktioner och exekvera dem på ett liknande sätt som i GHCi. 
Vi kommer att integrera jQuery \citep{jquery} för att få unisont stöd över samtliga webbläsare. JQuery kommer även underlätta arbetet med att skapa ett enkelt och stilrent interaktivt gränssnitt.

\subsubsection{Avgränsningar}
Att tolka Haskell i Javascript är inget trivialt projekt och därför kommer inte hela Haskell att implementeras. 
Endast en delmängd av Haskell98 specifikationen kommer att implementeras. De delar som prioriterades är
        \begin{enumerate}
            \item{lambda-funktioner, namngivna funktioner}
            \item{typer, generella typer, algebraiska datatyper}
            \item{typklasser}
            \item{pattern matching}
            \item{Guards}
            \end{enumerate}
Med dessa delar implementerade kan de flesta enklare haskellprogram köras och borde vara tillräckligt för det stora flertalet nybörjare. Man ska komma ihåg att detta projekt inte kommer resultera i något som ska ses som en ersättning till att använda vanliga haskellkompilatorer, såsom GHC och Hugs, utan en snabbare inkörsport för att lära sig Haskell. Därför anser vi att detta är en bra kompromiss som gagnar Haskellcommunityn mest. % :D :D: D

\FloatBarrier 
\newpage
%\input{kapitel/teori}\FloatBarrier
%\newpage
\section{Metod} 

Nedan följer en beskrivning av de arbetsmetoder vi använt oss utav och de mjukvaror och kodbibliotek som vi använt oss utav i projektet. 

\subsection{Arbetsmetodik}

% modulbaserat arbete..
Under planeringsstadiet upptäcktes tidigt att projektet kunde med enkelhet delas upp i tre separata moduler; parser, interpretator och typcheckare. Dessa tre moduler intergrerar enbart med varandra genom det abstrakta syntaxträdet. Detta medför att det är väldigt lätt att utveckla de olika delarna helt frånskilt från varandra. Figur 1 visar hur denna interaktion mellan de olika modulerna är tänkt att gå till. Man ser även att webbläsaren kommunicerar genom ett Javascript API och det abstrakta syntaxträdet och inte direkt med de olika komponenterna. 

\begin{figure}[H]
    \begin{center}
        \includegraphics[width=1.0\textwidth]{image1.png}
        \caption{Överblick över tolkens struktur och interaktion}
    \end{center}
\end{figure}


Arbetssättet präglades utav en iterativ utvecklingsmetodik med korta utvecklingscyklar. Arbetet delades upp med huvudansvarstagande över var sin modul. Arbetet skedde dock framförallt i samlad grupp för att snabbt kunna delge information om vad som behövde implementeras för att samverkan mellan de olika modulerna skulle fungera friktionsfritt.

\subsection{Kodstandard} 
För att få konsistens i koden och för att underlätta att olika utvecklare kan läsa och arbeta på koden samtidigt har vi utformat en intärn kodstandard som alla ska följa.
När en commit görs måste denna standard följas.
% MOOOAARRR!! 

\subsection{Versionshantering} 
Ett problem som alla mjukvaruprojekt av icke trivial storlek är att hantera den stora mängden filer, och distrubera uppdaterade kopior till samtliga utvecklare att arbeta på.
För att lösa detta problemet brukar man använda sig utav en versionshanteringsmjukvara. 

Under de första veckorna av projektet användes SVN. Valet berodde på att det var det som alla i medlemmar i projektet hade erfarenhet från tidigare. Tyvärr har SVN vissa problem när det kommer till att skapa nya förgreningar och sedan sammanfoga dem. Därför gick valet till att använda sig utav Git. Git är designat från grunden för att på ett enkelt sätt skapa nya och slå samman förgreningar under utvecklingens gång. Vi kunde därmed skapa en förgrening för varje modul och under arbetets gång sammanlänka allas arbeten på ett effektivt sätt. 

\subsection{Javascript} 
Javascript \citep{javascript} är ett programmeringsspråk som framförallt används på klientsidan på webben. Javascript är ett dynamiskt objektorienterat skriptspråk.

Javascript är det programmeringsspråk som används uteslutande i detta projektet för att skriva haskelltolken och interpretera Haskell.

\subsection{Kodbibliotek}

\emph{Standing on the Shoulder of Giants}

Detta projektet följer en fin tradition inom datorvetenskapen att om ett problem redan är löst så ska det inte behöva lösas igen. Att återuppfinna hjulet varje gång är både tidsödande och onödigt. 
Därför har ett antal kodbibliotek används i projektet. 
Genom att använda dessa kodbibliotek kan fokus läggas på implementeringen av de kärnområden som projektet behandlar.
Nedan följer en kort beskrivning av de olika kodbibliotek som vi använt i projektet.

 \subsubsection{JSparse}  

 %% Johan kanske kan skriva nåogt mer och vettigt här?
Parsern implementeras med hjälp av ett parser combinator bibliotek kallat JSParse.
En parser combinator använder sig utav olika regler som man kan kombinera för att skapa komplexa parsers. 
Denna parser combinator ger oss möjlighet att på ett enkelt sätt
implementera den del av  Haskells syntax som inte är context free.

\subsubsection{JQuery} 

JQuery är ett öppet kodbibliotek till Javascript som är dubeellicenserat under MIT License och GPL version 2.  
JQuery är designat för att underlätta för utvecklare att modifiera DOM-träd, HTML, och göra asynkrona javascript-anrop.

JQuery används i projektet för att få enkelt cross browser stöd utan att behöva tänka på det. 
JQuery ger även möjlighet att skapa ett enkelt och stilrent interaktivt gränssnitt utan att behöva göra allt från grunden.


% \subsection{Testning} 
% Vi gör inga fel så vi testar inte.. *skämt* *....seriously, we do not test..* ... yes, seriously 





\FloatBarrier
\newpage
\section{Resultat}
todo

\subsection{Parser} 
%TODO

\subsection{Abstrakt syntaxträd} 
%TODO

\subsection{Typcheckare} 
%TODO


\subsection{HIJi}


HIJi erbjuder användaren ett GHCi-liknande gränssnitt. HIJi fungerar som en fasad in i programmet. 
HIJi tar input genom att användaren skriver in Haskellkod som därefter tolkas av parsern och slutligen evalueras uttrycket. Resultatet av uttrycket visas på raden under.

HIJi är skapat för att likna GHCi i så stor utsträckning som möjligt. Det finns väldigt goda anledningar till att göra detta. Dels är GHCi ett mycket kompetent verktyg när man programmerar i Haskell. Att kolla upp funktionsdeklarationer och testa kodfragment är något som varje professionell haskellprogrammerare gör varje dag. Genom att efterlikna GHCi så kommer användare känna igen sig när de tar steget från HIJi till GHCi. Det blir för dem ett naturligt steg och kortar inlärningströskeln avsevärt. Även för haskellprogrammerare som är väl införstådda i GHCi's möjligheter blir det lättare att använda sig utav HIJi, de behöver inte fundera hur verktyget ska användas.
% Dock finns det vissa nackdelar med ett terminalliknande gränssnitt. 

\begin{figure}
    \begin{center}
        \includegraphics[width=1\textwidth]{hiji_screen3.png}
        \caption{HIJi användargränssnitt}
    \end{center}
\end{figure}

%Byt ut denna bilden om hiji uppdateras!
Bilden ovan visar hur HIJi ser ut för användaren. De första raderna visar precis som i GHCi vilka moduler som för närvarande är laddade. I detta exemplet är det en modul laddad med namnet Prelude. Därefter följer en prompt där användaren fritt kan skriva in egna funktioner. I exemplet har användaren använt sig utav en av de inbyggda funktionerna i Prelude och en lambda-funktion.

%TODO get source from cp-book at home

HIJi är tänkt att erbjuda användaren liknande möjligheter som GHCi. 

% TODO skriv om detta
Fördelen med HIJi framför GHCi är att användaren ej behöver ladda ner den tunga GHC-kompilatorn på sin personliga dator för att testa enkla Haskelluttryck direkt i webbläsaren.
Nackdelar gentemot GHCI är att HIJi är en nedbandat version utav GHCi. HIJi kan bara evaluera enklare uttryck. Det finns i dagsläget inga möjligheter att ladda upp hela Haskell-filer för att köra dem. Att som i GHCi på ett enkelt sätt kolla upp vilka typer en funktion stöds ej.

Ett stort problem för alla webbutvecklare idag är att de idag marknadsledande webbläsarna tolkar Javascript på olika sätt. Det har därför kommit fram en rad olika kodbibliotek för att lösa detta problemet. Ett av dessa är JQuery som HIJi använder sig utav för att få ett unisont stöd på alla moderna webbläsare. 


\FloatBarrier
\newpage
\section{Diskussion}
Vi har skapat en javascriptapplikation som kan parsa, typchecka och interpretera stora delar av Haskell 98. Det som saknas är fullständigt stöd för typklasser där stödet endast finns i typcheckaren och parsern men fortfarande behöver implementeras i interpretatorn. Detta handlar främst om förmågan att välja rätt instanser av typklasser vid applicering av överlagrade funktioner.

Sett till planeringen har vi lyckats uppfylla alla milstolpar utom typklasser, dock inte enligt den ordning och tidsplan som ursprungligen planerades. 
Vi insåg att det var enklast att utveckla parsern, typcheckaren och interpertatorn parallelt och bestämma individuellt vad som skulle implementeras och i 
vilken ordning för att senare, oftast en gång i veckan, samordna och implementera det som behövdes i flera delar.

Vi har inte implementerat NPlusK-pattern i parsern och då de är borttagna i Haskell 2010 \citep{haskell2010} känner vi att det inte behövs.

\subsection{Val av språk för implementering}
Tidigt i planeringen tvingades vi välja vilket språk vår implementation skulle
bestå av. Det första alternativet var att först skriva en kompilator för att
kompilera Haskell till Javascript. Därefter skulle så kallad boot-strapping
tillämpas där kompilatorn används för att kompilera sig själv till
Javascript. Eftersom det redan finns parsers och typcheckare för Haskell
skrivna i Haskell skulle projektet mestadels handla om att finna en lämplig
målrepresentation för Haskell i javascriptkod och sedan implementering av en
kodgenerator för denna. Haskells statiska typcheckning och referentiella
transparens skulle också förenkla verifiering av projektets komponenter.

Dock finns även en del nackdelar med en sådan implementation. Utan särskilda
optimeringar i kompilatorn skulle en målrepresentation nödvändigtvis innehålla
strukturer liknande dem man finner i en interpretator för ett lat evaluerat
språk. Det är ett rimligt antagande att sådana optimeringar på grund av sin komplexitet vore alltför stora för att genomföra i den aktuella tidsramen. Därför anser vi att denna typ av implementation lämpar sig bäst inom
ramarna för ett existerande kompilatorprojekt såsom GHC där nödvändiga optimeringar redan finns innbyggda.

Det andra alternativet var att skriva allt i Javascript och detta är den
implementationsstrategi som vi till slut bestämde oss för.
En fördel med en sådan implementation är att integrering med annan
javascriptkod blir enkel.
Det har också fördelen att i en interpretator bevaras kodstrukturen och en framtida interaktiv läromiljö får tillgång till mellanliggande körningsdata.
Det första alternativet skulle däremot kräva ett
särskilt integrationslager för att få samma möjligheter.


%stor del av vår projekts potentiella användare redan har viss erfarenhet av Javascript och liknande språk och att användande av vårt bibliotek därför blir enklare för dessa än vad motsvarande haskellimplementation skulle bli. Eftersom detta alternativ krävde att vi själva implementerar parser, typcheckare och interpreterare ansåg vi också att det gav oss större möjligheter till lärande. 

\subsection{Framtida förbättringar}

Syftet med projektet, att skapa en fungerande Haskelltolk i javascript, har vi lyckats implementera om man tar hänsyn till de avgränsningar som är uppsatta. Dock om man ser till motivationen bakom projektet, att vår Haskelltolk ska kunna användas som grund för att skapa en webbaserad interaktiv läroplattform, så finns det fortfarande mycket kvar att utveckla. Framförallt handlar det om att göra Haskelltolken och HIJi mer lättanvänd för nybörjare inom funktionell programmering.

I parsern har vi identifierat två förbättringsmöjligheter. För det första, hjälpsamma och förklarande felmeddelanden är en viktigt del av ett utvecklingsverktyg och det generars för tillfället inte av parsern. 
Om parsern stöter på ett fel rapporterar den endast att ett fel har inträffat och avslutar parsningsprocessen. 
Att förbättra dessa felmeddelanden med exempelivs rad- och kolumnnummer och specifik information om vad för fel som har inträffat skulle göra parsern mer användbar.
För att implentera detta behöver man kombinera steg 1 och 2 i parsningen för att rad- och kolumn-nummer ska bevaras korrekt då borttagning av nästlade kommentarer kan påverka dessa.
JSParse behöver modifieras så att det rapporterar var ett fel uppstod och i vilken parser.

För det andra, konverteringen av icke kontextfri Haskellkod till kontextfri kan förbättras 
för att klara av att expandera måsvingar i \emph{[x | let x = 5]}. 
För att klara av detta behövs en parser som räknar antal måsvingar, paranteser, 
komman och hakparanteser efter \emph{let} och avgöra när det är korrekt att sätta in avslutande måsvingar.

Även i HIJi finns det förbättringar att göra.
Det som framförallt behöver utvecklas är, för det första, erbjuda en interaktiv tutorial där användaren får instruktioner vad som ska skrivas in i HIJi. Om användaren skriver in rätt uttryck fortsätter tutorialen till nästa nivå.
För det andra, visa typinformation från funktioner genom att hålla musen över funktionsnamnet.
Och tillsist, kunna stega igenom ett program eller funktion för att kunna se vad som händer i varje evalutionssteg. 
\FloatBarrier
\newpage
\section{Slutsatser}

\subsection{Framtida förbättringar}


\subsubsection{Parser}
Hjälpsamma och förklarande felmeddelanden är en viktigt del av ett utvecklingsverktyg och genereras för tillfället inte av parsern. 
Om parsern stöter på ett fel rapporterar den att ett fel har inträffat och avslutar parsningsprocessen. 
Att förbättra dessa felmeddelanden med exempelivs rad- och kolumnnummer och specifik information om vad för fel som har inträffat skulle göra parsern mer användbar.

För att implentera detta behöver steg 1 och 2 kombiners för att rad- och kolumn-nummer ska bevaras korrekt då borttagning av nästlade kommentarer kan påverka dessa. 
JSParse behöver modifieras så att det rapporterar var ett fel uppstod och i vilken parser.

Konverteringen av icke kontextfri Haskellkod klarar inte av att expandera måsvingar i \empth{[x | let x = 5]},
för att lösa det behöver parsern modifieras så att den räknar antalet måsvingar, paranteser, komman och hakparanteser efter
\emph{let} och avgör när det är korrekt att sätta in avlsutande måsvingar.

\subsubsection{HIJi}
HIJi är tänkt som ett webbaserat verktyg för nybörjare att komma igång med funktionell programmering. Ur det perspektivet når HIJi ännu inte upp till de krav som en nybörjare kan förvänta sig. Avsaknaden av interaktivitet är den i dagsläget största nackdelen. Här är de förbättringar som vi anser vara nödvändiga för att nybörjare ska kunna använda sig utav HIJi.

Erbjuda en interaktiv tutorial där användaren får instruktioner vad som ska skrivas in i HIJi. Om använaren skriver in rätt uttryck fortsätter tutorialen till nästa nivå.

Få ut typinformation ur funktioner genom att hålla musen över funktionsnamnet. 
\FloatBarrier
\newpage
% \input{Method}\FloatBarrier
% \input{Results}\FloatBarrier
% \input{Conclusions}
% \input{Recommendations}

% And the bilbiography saved as mybib.bib
\bibliographystyle{plainnat}
\bibliography{kallor} 

% Appendices
\appendix
\appendix
\section{Bidragsrapport}
Alla gruppmedlemmar var delaktiga i likvärdig utsträckning under planeringsfasen. Större delen av projektet har genomdrivits gemensamt men varje person har haft en inriktning enligt projektets olika delar. Johan Gustafsson har haft ansvar över parsern, Mikael Bung har ansvarat för interpretatorn och det abstrakta syntaxträdet, Mattis Jeppsson för typcheckaren och Adam Bengtsson för HIJi. Rapporten har skrivits enligt samma uppdelning men med ett övergripande ansvar av Adam Bengtsson. Arbetet har i så stor utsträckning som möjligt skett gemensamt och vi har därför både bidragit till och tagit del av varandras respektive ansvarsområde. 
\FloatBarrier
\end{document}

